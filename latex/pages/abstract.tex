\chapter{\abstractname}

This thesis aims to identify potential improvements for B+ tree implementations using \ac{XAI} on \ac{ML} models. It starts by generating a dataset using the \ac{ycsb} benchmark, preprocessing the results, training \ac{ML} models, and finally analyzing the results of \ac{XAI} algorithms. Through this process, the thesis delves into various stages of the data science life cycle. Additionally, pattern recognition algorithms, Apriori and association rules, are described and used, which further verify some insights found by the \ac{XAI} approach.
\\\\
Building on the work of Müller et al. \parencite{mueller2024} the findings of this thesis contribute to our understanding of B+ trees by highlighting the impact of data skewness, size of values, number of records and the page size of leaf nodes on the performance. Beyond the domain of B+ trees, this thesis, concludes that the currently available \ac{XAI} algorithms can be utilized to enhance the interpretability of outputs originating from \ac{ML} models.   


